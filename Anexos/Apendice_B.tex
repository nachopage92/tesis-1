\chapter{Incorporación de Bases de Datos de Alta Resolución}
%La utilización del modelo WRF con dominios anidados hasta resoluciones del orden de los metros, exige la implementación de bases de datos no nativas del software para la información estática (orografía y uso de suelo) en los dominios.
%
%En este anexo se presenta el mapeo de las bases de datos descritas en las Tablas \ref{tab:05_caract_hov} y \ref{tab:05_caract_bol}, y una breve explicación acerca de cómo se manejaron las bases de datos.
%
%Con respecto a la manipulación de los datos, esta se realizó a través un software GIS (\emph{Geographic Information System}) y la librería GDAL para la conversión de la extensión de los archivos. En particular, se utilizó el programa QGIS que es gratuito y de código libre. Los trabajos hechos incluyen:
%\begin{itemize*}
%	\item Conversión de los datos Corine (CLC12) al formato de clasificación USGS24 y su correspondiente transformación a formato binario.
%	\item Transformación del datúm de la base de datos orográfica entregada por la campaña de medición de Bolund a WGS84.
%	\item Creación de la base de datos para el uso de suelo en Bolund según lo declarado por la campaña \citep{3d4285ac04444eb3b9775baf9af052c6}.
%	\item Refinamiento de los datos CLC12 (hecho de manera manual) para ajustar al contorno de la colina de Bolund.
%	\item Ajuste de alturas de la base de datos orográfica de Bolund para su correcta implementación en WPS (fijar la altura de agua en $z=0$).
%	\item Transformación de todos los datos a formato binario para la lectura del WPS.
%\end{itemize*}
%
%A continuación se presentan algunos ejemplos de códigos simples desarrollados para las tareas descritas.
%
%Código en QGIS para la transformación de CLC12 a USGS24 según \cite{Pineda2004}:
%\lstset{style=consola}
%\begin{lstlisting}
%("test@1"<=11)*1+("test@1"=12)*2+("test@1"=13)*3+("test@1"=14)*3+
%("test@1"=15)*6+("test@1"=16)*6+("test@1"=17)*6+("test@1"=18)*2+
%("test@1"=19)*6+("test@1"=20)*6+("test@1"=21)*6+("test@1"=22)*6+
%("test@1"=23)*11+("test@1"=24)*14+("test@1"=25)*15+("test@1"=26)*7+
%("test@1"= 27)*9+("test@1"=28)*9+("test@1"=29)*9+("test@1"=30)*19+
%("test@1"=31)*19+("test@1"=32)*19+("test@1"=33)*19+("test@1"=34)*24+
%("test@1"=35)*17+("test@1"=36)*17+("test@1"=37)*17+("test@1"=38)*17+
%("test@1"=39)*17+("test@1"=40)*16+("test@1"=41)*16+("test@1"=42)*16+
%("test@1"=43)*16+("test@1">=44)*16
%\end{lstlisting}
%Código GDAL para la generación de uso de suelo en Bolund a partir de su orografía:
%\begin{lstlisting}
%gdal_calc.py -A bolund_rought_displaced_wgs84.tif --outfile=result.tif 
%--calc="16*(A<0.01)+2*(A>0.01)"
%\end{lstlisting}
%Código GDAL para la conversión del uso de suelo de Bolund de GeoTIFF a binario:
%\begin{lstlisting}
%gdal_translate -of ENVI -ot Int16 result.tif BOLUND_LANDUSE.bil
%\end{lstlisting}
%Código GDAL para correjir los valores indefinidos generados por la rotación del datum:
%\begin{lstlisting}
%gdal_calc.py -A BOLUND_LANDUSE.bil --outfile=BOLUND_LANDUSE2.bil 
%--calc="(A==32767)*16+(A<32767)*A"
%\end{lstlisting}
%
%Finalmente las bases de datos introducidas de orografía y uso de suelo se ven en el modelo como las Figuras \ref{fig:dominios_hov} y \ref{fig:dominios_bol}. Notar que el color amarillo para la categoría de uso de suelo significa la incorporación manual del $z_0$ según la información presente en las publicaciones correspondientes para cada dominio.
%\newpage
%\vspace*{\fill}
%\begin{figure}[H]
%	\centering
%	\includegraphics[width=0.25\linewidth,page=1,trim={2cm 6.5cm 1cm 3.5cm},clip]{Imagenes/05/hov_domain.pdf}%
%	\includegraphics[width=0.25\linewidth,page=2,trim={2cm 6.5cm 1cm 3.5cm},clip]{Imagenes/05/hov_domain.pdf}%
%	\includegraphics[width=0.25\linewidth,page=3,trim={2cm 6.5cm 1cm 3.5cm},clip]{Imagenes/05/hov_domain.pdf}%
%	\includegraphics[width=0.25\linewidth,page=4,trim={2cm 6.5cm 1cm 3.5cm},clip]{Imagenes/05/hov_domain.pdf}%
%	
%	\bigskip
%	\includegraphics[width=0.25\linewidth,page=5,trim={2cm 6.5cm 1cm 3.5cm},clip]{Imagenes/05/hov_domain.pdf}%
%	\includegraphics[width=0.25\linewidth,page=6,trim={2cm 6.5cm 1cm 3.5cm},clip]{Imagenes/05/hov_domain.pdf}%
%	\includegraphics[width=0.25\linewidth,page=7,trim={2cm 6.5cm 1cm 3.5cm},clip]{Imagenes/05/hov_domain.pdf}%
%	\includegraphics[width=0.25\linewidth,page=8,trim={2cm 6.5cm 1cm 3.5cm},clip]{Imagenes/05/hov_domain.pdf}%
%	
%	\bigskip
%	\includegraphics[width=0.25\linewidth,page=9,trim={2cm 6.5cm 1cm 3.5cm},clip]{Imagenes/05/hov_domain.pdf}%
%	\includegraphics[width=0.25\linewidth,page=10,trim={2cm 6.5cm 1cm 3.5cm},clip]{Imagenes/05/hov_domain.pdf}%
%	\includegraphics[width=0.25\linewidth,page=11,trim={2cm 6.5cm 1cm 3.5cm},clip]{Imagenes/05/hov_domain.pdf}%
%	\includegraphics[width=0.25\linewidth,page=12,trim={2cm 6.5cm 1cm 3.5cm},clip]{Imagenes/05/hov_domain.pdf}%
%	
%	\bigskip
%	\includegraphics[width=0.25\linewidth,page=13,trim={2cm 6.5cm 1cm 3.5cm},clip]{Imagenes/05/hov_domain.pdf}%
%	\includegraphics[width=0.25\linewidth,page=14,trim={2cm 6.5cm 1cm 3.5cm},clip]{Imagenes/05/hov_domain.pdf}%
%	
%	\caption{Orografía (MSNM) y uso de suelo (categoría USGS24) de alta resolución para cada uno de las mallas anidadas (d01-d07) en Høvsøre.}
%	\label{fig:dominios_hov}
%\end{figure}
%\vspace*{\fill}
%\newpage
%\vspace*{\fill}
%\begin{figure}[H]
%	\centering
%	\includegraphics[width=0.25\linewidth,page=1,trim={2cm 6.5cm 1cm 3.5cm},clip]{Imagenes/05/bol_domain.pdf}%
%	\includegraphics[width=0.25\linewidth,page=2,trim={2cm 6.5cm 1cm 3.5cm},clip]{Imagenes/05/bol_domain.pdf}%
%	\includegraphics[width=0.25\linewidth,page=3,trim={2cm 6.5cm 1cm 3.5cm},clip]{Imagenes/05/bol_domain.pdf}%
%	\includegraphics[width=0.25\linewidth,page=4,trim={2cm 6.5cm 1cm 3.5cm},clip]{Imagenes/05/bol_domain.pdf}%
%	
%	\bigskip
%	\includegraphics[width=0.25\linewidth,page=5,trim={2cm 6.5cm 1cm 3.5cm},clip]{Imagenes/05/bol_domain.pdf}%
%	\includegraphics[width=0.25\linewidth,page=6,trim={2cm 6.5cm 1cm 3.5cm},clip]{Imagenes/05/bol_domain.pdf}%
%	\includegraphics[width=0.25\linewidth,page=7,trim={2cm 6.5cm 1cm 3.5cm},clip]{Imagenes/05/bol_domain.pdf}%
%	\includegraphics[width=0.25\linewidth,page=8,trim={2cm 6.5cm 1cm 3.5cm},clip]{Imagenes/05/bol_domain.pdf}%
%	
%	\bigskip
%	\includegraphics[width=0.25\linewidth,page=9,trim={2cm 6.5cm 1cm 3.5cm},clip]{Imagenes/05/bol_domain.pdf}%
%	\includegraphics[width=0.25\linewidth,page=10,trim={2cm 6.5cm 1cm 3.5cm},clip]{Imagenes/05/bol_domain.pdf}%
%	\includegraphics[width=0.25\linewidth,page=11,trim={2cm 6.5cm 1cm 3.5cm},clip]{Imagenes/05/bol_domain.pdf}%
%	\includegraphics[width=0.25\linewidth,page=12,trim={2cm 6.5cm 1cm 3.5cm},clip]{Imagenes/05/bol_domain.pdf}%
%	
%	\bigskip
%	\includegraphics[width=0.25\linewidth,page=13,trim={2cm 6.5cm 1cm 3.5cm},clip]{Imagenes/05/bol_domain.pdf}%
%	\includegraphics[width=0.25\linewidth,page=14,trim={2cm 6.5cm 1cm 3.5cm},clip]{Imagenes/05/bol_domain.pdf}%
%	
%	\bigskip
%	\includegraphics[width=0.25\linewidth,page=15,trim={0cm 6.5cm 1cm 3.5cm},clip]{Imagenes/05/bol_domain.pdf}%
%	\includegraphics[width=0.25\linewidth,page=16,trim={0cm 6.5cm 1cm 3.5cm},clip]{Imagenes/05/bol_domain.pdf}%
%	
%	\caption{Orografía (MSNM) y uso de suelo (categoría USGS24) de alta resolución para cada uno de las mallas anidadas (d01-d08) en Bolund.}
%	\label{fig:dominios_bol}
%\end{figure}
%\vspace*{\fill}
