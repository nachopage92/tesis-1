\chapter{Metodología de la Investigación}

( Hablar sobre la metodología, como crear nubes, verificar calidad, explicar los test )

\section{Nube de aproximación}

Triangulación de Delaunay

\subsection{Criterio de creación de nube}

\subsection{Verificación de la calidad de la nube}
Para verificar la calidad de la nube se implementan dos pruebas, la primera consiste en verificar el cumplimiento de las condiciones de consistencia, es decir, (*****)

\begin{eqnarray} \label{eq:d0_consistency_conditions}
    \sum_{i=1}^{np} \phi_i - 1 = 0 \\
    \sum_{i=1}^{np} \phi_i \vec{x}_i = \vec{x}
\end{eqnarray}

\begin{eqnarray} \label{eq:d1_consistency_conditions}
    \sum_{i=1}^{np} \frac{\partial{\phi_i}}{\partial{\vec{x}}} = \vec{0} \\
    \sum_{i=1}^{np} \frac{\partial\{\phi_i\vec{x}_i\}}{\partial\vec{x}} - \textbf{Id} = \textbf{0} 
\end{eqnarray}

\begin{eqnarray} \label{eq:d2_consistency_conditions}
    \sum_{i=1}^{np} \frac{\partial^2{\phi_i}}{\partial x^2} = \textbf{0} \\
    \sum_{i=1}^{np} \frac{\partial^2\{\phi_i\vec{x}_i\}}{\partial\vec{x}^2} = \mathbb{M}_0 
\end{eqnarray}
donde $\mathbb{M}_0$ es un tensor nulo de orden 3

Mediante una aproximación de espacios convexo se verifica la calidad de la aproximación, la función de forma debiera aproximar :
\begin{eqnarray}
    \sum_{i=1}^{np} f(\vec{x}_i) \phi_i(\vec{x}) - f(\vec{x}) = 0 \\
    \sum_{i=1}^{np} \frac{\partial\{f(\vec{x}_i)\phi_i(\vec{x})\}}{\partial \vec{x}} - \frac{\partial f(\vec{x})}{\partial \vec{x}} = \vec{0} \\
    \sum_{i=1}^{np} \frac{\partial^2\{f(\vec{x}_i)\phi_i(\vec{x})\}}{\partial \vec{x}^2} - \frac{\partial^2 f(\vec{x})}{\partial \vec{x}^2} = \textbf{0}
\end{eqnarray}
Se utiliza la siguiente función como test:
\begin{equation}
    f(\vec{x}) = \vec{x} \cdot \vec{x}
\end{equation}
es decir, $x^2$ y $x^2+y^2$ en 1D y 2D respectivamente.


\subsection{Generación de nube}
Entre las alternativas para generar nubes de aproximación se encuentran (citar a rainald). En este trabajo se utilizó el algoritmo propuesto por Rainald (citar libro), rutina que crea una colección de puntos discretos cercanos previamente triangulados (Triangulación de Delaunay).

\section{Pruebas numéricas en espacios unidimensionales y bidimensionales}

\subsection{Test 1D 1}
cuya solución exacta es
\begin{eqnarray}
    u = \sin( 2 \pi x ) \\
    du = 2 \pi \cos( 2 \pi x )
\end{eqnarray}

\subsection{Test 1D 2}
cuya solución exacta es
\begin{eqnarray}
    u  = 1 - x^3 + exp(-100 x^2 )
    du = - 3 x^2 -200 x \exp(-100 x^2 )
\end{eqnarray}

\subsection{Test 1D 3, ElasticBar}
cuya solución exacta es
\begin{eqnarray}
    u = \frac{1}{E} ( \frac{1}{2} x - \frac{x^3}{6})
    du = \frac{( 1 -x^2 )}{2}
\end{eqnarray}

\subsection{Test 1D 4, RW}
cuya solución exacta es
\begin{eqnarray}
    u  = ( 1-x ) (\arctan(a*(x-xo))+\arctan(a*xo))
    du = ( 1-x )*\frac{a}{( 1 + a^2 (x-xo)^2 )} - \arctan(a*(x-xo)) + \arctan(a*xo)
\end{eqnarray}

\subsection{Test 1D 5, AxialTruss}
La ecuación governante es
\begin{equation}
    E A \frac{d^2 u}{d^2 x} + b(x) = 0
\end{equation}
y su término fuente es:
\begin{equation}
    b(x) = -(2.3 \pi)^2 \sin(2.3\pi x)
\end{equation}
donde su extremo izquierdo ($x=0$) se encuentra fijo y su extremo derecho ($x=1$) se le aplica una fuerza sinusoidal a lo largo del eje $x$, es decir:
\begin{equation}
    u|_{x=0} = 0
\end{equation}
\begin{equation}
    \begin{split}
        f = A \sigma_x|_{x=1} = E A \frac{d u}{d x} |_{x=1} = -2.3 \pi \cos(2.3 \pi x) , \\
        \mbox{ o bien, } \hspace{0.5cm} \frac{d u}{d x} |_{x=1} = -2.3 \pi \cos(2.3 \pi x)
    \end{split}
\end{equation}

cuya solución exacta es
\begin{eqnarray}
    u = -sin( 2.3 \pi x ) 
    du = - 2.3 \pi \cos( 2.3 \pi x )
\end{eqnarray}

\subsection{Test 1D 6, Waveprop}
cuya solución exacta es
\begin{eqnarray}
    u = \frac{ sin( \sqrt{\lambda} x) }{ sin( \sqrt{\lambda} ) }
    du = \sqrt{\lambda} \frac{ cos( \sqrt{\lambda} x )} {( sin( \sqrt{\lambda} ))}
\end{eqnarray}

\subsection{Test 1D 1, T1}
