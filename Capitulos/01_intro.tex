\chapter{Introducción}

%\begin{figure}[h]
%	\centering
%	\includegraphics[width=0.9\linewidth,trim={1.4cm 28cm 15cm 3.4cm},clip]{Imagenes/01/explo}
%	\caption{Interfaz online del explorador eólico de la Universidad de Chile.}
%	\label{fig:01_explorador}
%\end{figure}

%\begin{figure}
%	\begin{minipage}{0.5\linewidth}
%		\centering
%		(a)
%	\end{minipage}
%	\begin{minipage}{0.5\linewidth}
%		\centering
%		(b)
%	\end{minipage}
%	
%	\begin{minipage}{0.5\linewidth}
%		\centering
%		\includegraphics[width=0.9\linewidth,page=3,trim={6cm 12.2cm 6cm 9.5cm},clip]{Imagenes/01/descrp}
%	\end{minipage}
%	\begin{minipage}{0.5\linewidth}
%		\centering
%		\includegraphics[width=0.7\linewidth,trim={0cm 0cm 0cm 0cm},clip]{Imagenes/01/prototipo}
%	\end{minipage}
%	\caption{Detalle del proyecto FONDEF ID16I10105. (a) Célula del sistema experimental de medición. (b) Prototipo en el laboratorio.}
%	\label{fig:01_detalle_fondef}
%\end{figure}


%\begin{figure}[h!]
%	\centering
%	\includegraphics[width=0.82\linewidth,page=5,trim={3cm 2cm 2.3cm 3cm},clip]{Imagenes/01/descrp}
%	\caption{Esquema de la sonda FONDEF ID16I10105.}
%	\label{fig:01_sonda}
%\end{figure}


\section{Hipótesis}
La función de forma obtenida mediante la optimización del función de entropía representa una mejor opción respecto a la métodología de minimos cuadrados ponderados fijos. 

\section{Objetivos}
\subsubsection{Objetivo Principal}
\begin{itemize*}
    \item Implementar un código que permita utilizar distintas funciones de forma y comparar sus resultados
\end{itemize*}
\subsubsection{Objetivos Secundarios}
\begin{itemize*}
    \item Hallar alternativas de soluciones prácticas que permita subsanar los problemas que presentan los esquemas de colocación directo, particulamente en la imposición de condiciones naturales o de Neumann
\end{itemize*}
\newpage
\section{Estructura del Documento}
La estructura de esta tesis se organiza de la siguiente manera:
\begin{itemize*}
	\item Cap. 2: Se exponen los últimos avances de investigación asociados al uso de metodologías MeshFree en la resolución de problemas de interés de Ingeniería. 
        \item Cap. 3: Sienta las bases conceptuales del presente trabajo, entre sus tópicos se encuentra la formulación MeshFree, los esquemas de colocación puntual. Se introduce la función de forma FWLS (Fixed Weighted Least Squares) exponiendo sus ventajas y desventajas que buscan ser subsanas por el presente método propuesto. 
	\item Cap. 4: Se presenta formalmente el funcional de Máxima Entropía, su origine como estimador de la incertimbre y su implementación como función de forma.
	\item Cap. 5: Se explica el funcionamiento del código implementado. Se presentan soluciones a los problemas prácticos presentes en la implementación MeshFree de colocación, Se exponen ensayos numéricos con solución teórica conocida para evaluar el desempeño del programa computacional. Se implementan pruebas en 1D y en 2D donde se consideran tanto condiciones de borde esenciales o de Dirichlet como condiciones naturalez o de Neumann
	\item Cap. 6: Se presentan y analizan detalladamente los resultados mas relevantes.
	\item Cap. 7: Conclusiones, trabajo futuro y propuestas de mejora para el trabajo/método implementado.
\end{itemize*}
