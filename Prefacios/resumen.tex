\prefacesection{Resumen}
Con el fin de proponer una nueva metodología para la correcta predicción del recurso viento en terreno complejo para zonas muy localizadas, se llevaron cabo una serie de simulaciones numéricas meteorológicas multiescala con datos reales utilizando el software WRF y una clausura LES para la turbulencia.
El acoplamiento de las meso y microescala se logra a través de la técnica de dominios anidados hasta llegar a una resolución de aproximadamente 2 [m]. Para corregir las desviaciones propias de una simulación numérica, se propuso mejorar los resultados utilizando un esquema de asimilación de datos cuatridimensional en el dominio mas fino.

Se presentan resultados para 4 casos. Los primeros dos casos corresponden a una simulación real en el sitio de pruebas de turbinas en Høvsøre, Dinamarca, el cual es un terreno cuasi-plano ampliamente estudiado. La primera simulación valida el acercamiento numérico y la segunda muestra la influencia de la asimilación de datos en la capa límite considerando 6 niveles de un mástil meteorológico ubicado en el centro del dominio.

Las siguientes dos simulaciones corresponden a la aplicación de la misma metodología en terreno complejo. En este caso se simula la colina de Bolund ubicada también en Dinamarca. Estas dos simulaciones ahora nos exponen el comportamiento del modelo para este caso, y la influencia de la asimilación de datos multipunto utilizando la información de 8 mástiles y en 3 niveles cercanos a la superficie.

\textcolor{red}{Los resultados obtenidos muestran que es posible obtener predicciones certeras y que rescaten el comportamiento turbulento del viento a las escalas simuladas y que además, la asimilación de datos mejora considerablemente esta predicción, dando pié a un uso operativo de los códigos utilizados.}

\paragraph{Palabras Clave} \emph{Simulación Multiescala, LES, Asimilación de Datos, WRF, Capa Limite Atmosférica, NWP, Turbulencia Atmosférica, Energía Eólica}