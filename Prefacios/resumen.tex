\prefacesection{Resumen}
Con el fin de lograr una correcta predicción del recurso viento en terreno complejo en zonas muy localizadas, se llevaron cabo una serie de simulaciones numéricas multiescala con datos reales utilizando WRF-LES a través de la técnica de dominios anidados hasta una resolución mínima de aproximadamente 2 [m]. Para corregir las desviaciones propias de una simulación numérica, se propuso mejorar los resultados utilizando un esquema de asimilación de datos en el dominio mas interior.

Se presentan resultados para 4 casos, los primeros dos casos corresponden a una simulación real en el sitio de pruebas de turbinas en Høvsøre, Dinamarca, el cual es un terreno cuasi-plano ampliamente estudiado. La primera simulación valida el acercamiento numérico y la segunda muestra la influencia de la asimilación de datos considerando 3 niveles de un mástil meteorológico ubicado en el centro del dominio.

Las siguientes dos simulaciones corresponden a la aplicación de la metodología en terreno complejo. En este caso se simula la colina de Bolund ubicada también en Dinamarca. Estas dos simulaciones ahora nos exponen el comportamiento del modelo para este caso, y la influencia de la asimilación de datos multipunto utilizando la información de 8 mástiles.

Los resultados obtenidos muestran que es posible obtener predicciones certeras y que rescaten el comportamiento turbulento del viento a las escalas simuladas y que además, la asimilación de datos mejora considerablemente esta predicción, dando pié a un uso operativo de los códigos utilizados.

%\begin{figure}[H]
%	\centering
%	\includegraphics[width=0.8\linewidth,trim={0cm 0cm 0cm 0cm},clip]{Imagenes/}
%	\caption{test.}
%	\label{fig:}
%\end{figure}